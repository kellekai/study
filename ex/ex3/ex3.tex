\documentclass[a4paper]{article}
\usepackage{mathtools}
\usepackage{amsmath}
\usepackage{amsthm}
\usepackage{amssymb}
\usepackage{framed}
\usepackage[makeroom]{cancel}
\usepackage{hyperref}

\newcommand{\pd}[3]{\left(\frac{\partial {#1}}{\partial {#2}}\right)_{#3}}
\newcommand{\pdd}[3]{\left(\frac{\partial^2 {#1}}{\partial {#2}\partial {#3}}\right)}

\title{Exercise sheet 3}
\author{Kai Keller}
\date{\today}

\begin{document}
\maketitle
\tableofcontents

\section{problem 1 - Weiss-Curie Ferromagnetism}
Consider a System with the following Hamiltonian:
\begin{equation}
    H(s)=-\frac{J}{2N}\sum_i\sum_j s_i s_j -h\sum_i s_i \label{wchamiltonian}
\end{equation}

\subsection{(a, 2P)}
\paragraph{show:}

That its Partition function can be written as:
\begin{equation}\label{partitionfunction}
  Z(\beta,J,h)=\sqrt{\frac{N\beta J}{2 \pi}}\,\int_{-\infty}^{+\infty}\,dx\,e^{-N\beta Jx^2/2}\,C(x)
\end{equation} 
With:
\begin{equation}
    C(x)=\prod\limits_i\,\sum_{\{s_i\}}\,e^{\beta s_i(Jx+h)} \label{cvonx}
\end{equation}
Let's consider first the general case of the summation done in eq. \ref{cvonx}:
\begin{equation}
    \Omega=\prod\limits_i\,\sum_{s_{i[,j]}}\,f(s_{i[,j]})
\end{equation}
Where the index $j=1,\dots,M$ runs over the members of the configuration $s_i$, $i=1,\dots,N$. We may expand the expression to:
\begin{align}
    \Omega&=\prod\limits_i\,(f(s_{i[,1]})+\dots+f(s_{i[,M]})) \\
    &=(f(s_{1[,1]})+\dots+f(s_{1[,M]}))\times\dots\times(f(s_{N[,1]})+\dots+f(s_{N[,M])})
\end{align}
Which can be reordered to:
\begin{align}
    \Omega&=(f(s_{1[,1]}) \dots f(s_{N[,1]}))+\dots +(f(s_{1[,M]}) \dots f(s_{N[,M]})) \\
    &=\prod\limits_i f(s_{i[,1]})+\dots +\prod\limits_i f(s_{i[,M]})
\end{align}
Where the last expression corresponds to a sum over all possible configurations. Hence we may write:
\begin{equation}
    \Omega=\prod\limits_i\,\sum_{s_{i[,j]}}\,f(s_{i[,j]})=\sum_{\{\vec{s}\}}\,\prod\limits_i\,f(s_{i[,j]})
\end{equation}
Using this, we may write eq. \ref{partitionfunction} as:
\begin{align}
    Z(\beta,J,h)&=\sum_{\{\vec{s}\}}\sqrt{\frac{N\beta J}{2 \pi}}\,\int_{-\infty}^{+\infty}\,dx\,e^{-N\beta Jx^2/2}\,\prod\limits_i\,e^{\beta s_i(Jx+h)} \\
    &=\sum_{\{\vec{s}\}}\sqrt{\frac{N\beta J}{2 \pi}}\,\int_{-\infty}^{+\infty}\,dx\,e^{-N\beta Jx^2/2}\,e^{\beta (\sum_i\,s_iJx+\sum_i s_i h)}\\
    &=\sum_{\{\vec{s}\}}\sqrt{\frac{N\beta J}{2 \pi}}\,e^{\beta\sum_i s_i h}\int_{-\infty}^{+\infty}\,dx\,e^{-N\beta Jx^2/2}\,e^{\beta J\sum_i\,s_ix}
\end{align}
We perform a variable ransform $y=\sqrt{N\beta J}x\implies dx=dy/\sqrt{N\beta J}$. Now we may write:
\begin{equation}
    Z(\beta,J,h)=\sum_{\{\vec{s}\}}\sqrt{\frac{1}{2 \pi}}\,e^{\beta\sum_i s_i h}\int_{-\infty}^{+\infty}\,dx\,e^{-y^2/2-by}
\end{equation}
with:
\begin{align}
    b&=-\frac{\beta J}{\sqrt{N\beta J}}\sum_i\,s_i \\
    b^2&=\frac{\beta J}{N}\sum_i\,\sum_j\,s_i s_j
\end{align}
This is a general form of a gaussian integral with solution:
\begin{equation}
    \int_{-\infty}^{+\infty}\,dx\,e^{-y^2/2-by}=\sqrt{2 \pi}\,e^{b^2/2}
\end{equation}
This leads to:
\begin{align}
    Z(\beta,J,h)&=\sum_{\{\vec{s}\}}\,e^{\beta\sum_i s_i h+\frac{\beta J}{N}\sum_i\,\sum_j\,s_i s_j} \\
    &=\sum_{\{\vec{s}\}}\,e^{-\beta H(s)}
\end{align}
which is precisely the definition of the partition function.

\subsection{(a, 2P)}
\paragraph{Compute:}

$C(x)$ explicitely.

We have only 2 posible values for each $s_i$. We hence have:
\begin{align}
    C(x)&=\prod\limits_i\,\left( e^{\beta (Jx+h)} + e^{-\beta (Jx+h)} \right) \\
    &=\prod\limits_i\,2 \sin{(\beta (Jx+h))} \\
    &=\sinh{(\beta (Jx+h))}^N
\end{align}

\end{document}


