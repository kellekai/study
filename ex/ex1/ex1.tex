\documentclass[a4paper]{article}
\usepackage{amsmath}
\usepackage{amsthm}
\usepackage{amssymb}
\usepackage{framed}
\usepackage[makeroom]{cancel}
\usepackage{hyperref}

\newcommand{\pd}[3]{\left(\frac{\partial {#1}}{\partial {#2}}\right)_{#3}}
\newcommand{\pdd}[3]{\left(\frac{\partial^2 {#1}}{\partial {#2}\partial {#3}}\right)}

\title{Exercise sheet 2}
\author{Kai Keller}
\date{\today}

\begin{document}
\maketitle
\tableofcontents

\section{problem 1 - Thermodynamic Potentials and State Variables}
\subsection{(a, 2P)}
\paragraph{show:}
\begin{equation}
    \left(\frac{\partial f}{\partial u}\right)_v \, \left(\frac{\partial u}{\partial f}\right)_v = 1 
\end{equation} 
we may write:
\begin{equation}
    u=f^{-1}\left(f(u)\right)
\end{equation}
with that we can write for the differential
\begin{equation}
    du=df^{-1}=\left(\frac{\partial f^{-1}}{\partial f}\right)_v\,\left(\frac{\partial f}{\partial u}\right)_v\, du=\left(\frac{\partial u}{\partial f}\right)_v\,\left(\frac{\partial f}{\partial u}\right)_v\, du
\end{equation}
by comparing the l.h.s with the r.h.s we get:
\begin{framed}
    \begin{equation}\label{res:1a}
    \left(\frac{\partial u}{\partial f}\right)_v\,\left(\frac{\partial f}{\partial u}\right)_v=1 \quad \qed
\end{equation}
\end{framed}

\paragraph{show:}

\begin{equation}
    \left(\frac{\partial f}{\partial u}\right)_v\left(\frac{\partial u}{\partial v}\right)_f\left(\frac{\partial v}{\partial f}\right)_u=-1
\end{equation}

to start we write down the differential forms of $u$ and $f$:
\begin{align}
    df&=\left(\frac{\partial f}{\partial u}\right)_vdu+\left(\frac{\partial f}{\partial v}\right)_udv \\
    du&=\left(\frac{\partial u}{\partial f}\right)_vdf+\left(\frac{\partial u}{\partial v}\right)_fdv
\end{align}

Combining these two, we may write:
\begin{align}
    df=&\left(\frac{\partial f}{\partial u}\right)_v\left(\left(\frac{\partial u}{\partial f}\right)_vdf+\left(\frac{\partial u}{\partial v}\right)_fdv\right)+\left(\frac{\partial f}{\partial v}\right)_udv \\
    =&\underbrace{\left(\frac{\partial f}{\partial u}\right)_v\left(\frac{\partial u}{\partial f}\right)_v}_{=1 \textit{ (from eq. \ref{res:1a})}}df+\left(\frac{\partial f}{\partial u}\right)_v\left(\frac{\partial u}{\partial v}\right)_fdv+\left(\frac{\partial f}{\partial v}\right)_udv
\end{align}

Now we can reorder and multiply by $\left(\frac{\partial v}{\partial f}\right)_u$ and we get:
\begin{align}
    \left(\frac{\partial f}{\partial u}\right)_v\left(\frac{\partial u}{\partial v}\right)_fdv&=-\left(\frac{\partial f}{\partial v}\right)_udv\\
    \left(\frac{\partial v}{\partial f}\right)_u \left(\frac{\partial f}{\partial u}\right)_v\left(\frac{\partial u}{\partial v}\right)_fdv& = \underbrace{\left(\frac{\partial v}{\partial f}\right)_u \left(\frac{\partial f}{\partial v}\right)_u}_{=1}dv \\
    &=-1\, dv
\end{align}
By comparing the l.h.s. with the r.h.s. we get
\begin{framed}
    \begin{equation}
        \left(\frac{\partial f}{\partial u}\right)_v\left(\frac{\partial u}{\partial v}\right)_f\left(\frac{\partial v}{\partial f}\right)_u=-1 \quad \qed
    \end{equation}
\end{framed}

\subsection{(b, 1P)}\label{problem1}
\paragraph{Relate:}

The isochoric pressure change $\left( \frac{\partial p}{\partial T}  \right)_V$ to the standard response functions:
\begin{align}
    \alpha &= \frac{1}{V} \left( \frac{\partial V}{\partial T} \right)_p \label{alpha}\\ 
    \kappa &= -\frac{1}{V} \left( \frac{\partial V}{\partial p} \right)_T \label{kappat}
\end{align}

We start writing down the differentials for $p$ and $V$:
\begin{align}
    dp&=\left(\frac{\partial p}{\partial V}\right)_TdV+\left(\frac{\partial p}{\partial T}\right)_VdT \\
    dV&=\left(\frac{\partial V}{\partial p}\right)_Tdp+\left(\frac{\partial V}{\partial T}\right)_pdT
\end{align}
By multiplying $dV$ in the differential with $\left(\frac{\partial p}{\partial V}\right)_T$, we can combine $dp$ and $dV$ to:
\begin{equation}
    0=\left[ \left(\frac{\partial p}{\partial V}\right)_T\left(\frac{\partial V}{\partial T}\right)_p+\left(\frac{\partial p}{\partial T}\right)_V \right]dT
\end{equation}
and finally:
\begin{framed}
    \begin{equation}
        \left(\frac{\partial p}{\partial T}\right)_V=-\left(\frac{\partial V}{\partial p}\right)_T^{-1}\left(\frac{\partial V}{\partial T}\right)_p=\frac{\alpha}{\kappa}
    \end{equation}
\end{framed}
\subsection{(c,d 2P)}
\paragraph{Express:}
The differential forms of the caloric states $U(p,T)$ and $U(V,T)$, by standard response functions.

We write down the differentials for the caloric states:
\begin{align}
    dU&=\pd{U}{p}{T}dp+\pd{U}{T}{p}dT \label{dupt}\\
    dU&=\pd{U}{V}{T}dV+\pd{U}{T}{p}dT \label{duvt}
\end{align}
We will use the expression for Gibbs and Helmholtz free energy:
\begin{align}
    G=G(p,T)=U-TS+pV=Vdp-SdT \label{gfe} \\
    F=F(V,T)=U-TS=-pdV-SdT \label{hfe}
\end{align}
and the Maxwell relation    
\begin{equation}
    \pd{S}{V}{T} = \pd{p}{T}{V} \label{mrsvtopt}
\end{equation}
This relation holds since:
\begin{equation}
    \pd{S}{V}{T}=-\pdd{F}{T}{V}=-\pdd{F}{V}{T}=\pd{p}{T}{V} \quad \qed
\end{equation}
We want to express the partial derivatives in eq. \ref{dupt} and \ref{duvt}. 
\\${}$\\
\textbf{i:}$\pd{U}{V}{T}$
\\${}$\\
$V$ and $T$ are the natural variables of the Helmholtz free energy F. Using eq. \ref{hfe} yields:
\begin{equation}
    \pd{F}{V}{T}=\pd{U}{V}{T}-T\pd{S}{V}{T} \quad \text{// $\pd{T}{V}{T}=0$ obviously}
\end{equation}
comparing the partial derivatif of $F$ on the l.h.s. with eq. \ref{hfe} and replacing the second term of the r.h.s by its Maxwell relation (eq. \ref{mrsvtopt}), we may write:
\begin{align}
    -p &= \pd{U}{V}{T}-T\pd{p}{T}{V} \\
    \implies \pd{U}{V}{T}&=\underbrace{-p+T\pd{p}{T}{V}}_{\text{for part d}}=-p+T\frac{\alpha}{\kappa_T} \label{pduvt}
\end{align}
Where the last step follows from solution of part \ref{problem1}b.
\\${}$\\
\textbf{ii:}$\pd{U}{T}{V}$
\\${}$\\
We will again use $F$, since we have the same variables as before. Taking the partial derivative yields:
\begin{equation}
    \pd{F}{T}{V}=\pd{U}{T}{V}-S-T\pd{S}{T}{V}
\end{equation}
Now we want to find an expression for the last term on the r.h.s.. We can infer:
\begin{equation}
    \pd{S}{T}{V}=\pd{T}{S}{V}^{-1}=\pdd{U}{S}{S}^{-1}=\frac{1}{T}\theta_V^{-1}
\end{equation}
where,
\begin{equation}
    \theta_V:=\frac{1}{T}\pdd{U}{S}{S}_V
\end{equation}
Comparing the partial derivatif of $F$ on the l.h.s. with eq. \ref{hfe} we may write:
\begin{equation}
    \pd{U}{T}{V}=\frac{1}{\theta_V}=C_V \label{pdutv}
\end{equation}
\\${}$\\
\textbf{iii:}$\pd{U}{p}{T}$
\\${}$\\
We will look at Gibbs free energy, since $p$ and $T$ are it's natural variables. We may write:
\begin{equation}
    \pd{G}{p}{T}=\pd{U}{p}{T}-T\pd{S}{p}{T}+V+p\pd{V}{p}{T}
\end{equation}
Using the same strategy as from above, we can easily identify $\pd{G}{p}{T}=V$. The last term is related by eq. \ref{kappat} to the response function $\kappa_T$. So we only have to determine one term. We may write:
\begin{equation}
    \pd{S}{p}{T}=-\pdd{G}{T}{p}=-\pdd{G}{p}{T}=-\pd{V}{T}{p}=-V\alpha
\end{equation}
Where we applied eq. \ref{alpha} in the last step. Now we have:
\begin{equation}
    \pd{U}{p}{T}=(\kappa_Tp-\alpha T)V \label{pdupt}
\end{equation}
\\${}$\\
\textbf{iv:}$\pd{U}{T}{p}$
\\${}$\\
Since we have the same variables here as for the case iii, we will look again on the Gibbs free energy:
\begin{equation}
    \pd{G}{T}{p}=\pd{U}{T}{p}-S-T\pd{S}{T}{p}+p\pd{V}{T}{p}
\end{equation}
Using the same strategies as above, we may immediately write:
\begin{equation}
    -S=\pd{U}{T}{p}-S-T\pd{S}{T}{p}+pV\alpha
\end{equation}
The last term arises by usage of eq. \ref{alpha}. Let us identify the missing term:
\begin{equation}
    \pd{S}{T}{p}=\pd{T}{S}{p}^{-1}=\pdd{U}{S}{S}^{-1}=\frac{1}{T}\theta_p^{-1}
\end{equation}
where,
\begin{equation}
    \theta_p:=\frac{1}{T}\pdd{U}{S}{S}_p
\end{equation}
With this, we have:
\begin{equation}
    \pd{U}{T}{p}=\frac{1}{\theta_p}-PV\alpha=C_p-pV\alpha \label{pdutp}
\end{equation}
Collecting the expressions for the partial derivatives of $U$ from equations \ref{pduvt}, \ref{pdutv}, \ref{pdupt} and \ref{pdutp}, we may finally write:
\begin{framed}
    \begin{align}
        dU_{V,T}&=\left( T\frac{\alpha}{\kappa_T}-p \right) dV + C_V dT \\
        dU_{p,T}&=\left( \kappa_T -\alpha T \right) V dp+\left( C_p -pV\alpha  \right) dT
    \end{align}
\end{framed}

\end{document}


