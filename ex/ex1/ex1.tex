\documentclass[a4paper]{article}
\usepackage{amsmath}
\usepackage{framed}
\usepackage[makeroom]{cancel}

\title{Exercise sheet 1}
\author{Kai Keller}
\date{\today}

\begin{document}
\maketitle
\tableofcontents

\section{problem 1 - Gaussian Integrals}
\subsection{(a)}
\begin{equation}\label{eq:stdgauss}
I=\int_{-\infty}^{\infty}\,dx\,e^{-\frac{x^2}{2}}
\end{equation} 
The trick is to compute $I^2$ instead of $I$. We get:
\begin{equation}
I^2=\int_{-\infty}^{\infty}\,\int_{-\infty}^{\infty}\,dx\,dy\,e^{-\frac{1}{2}(x^2+y^2)}
\end{equation}
this we may express in polar choordinates:
\begin{equation}
I^2=\int_{0}^{\infty}\,\int_{-\pi}^{\pi}\,dr\,d\theta\,re^{-\frac{1}{2}r^2} = \dots
\end{equation}
performing a variable transform from $r^2 \rightarrow u$ and perfoming the integration for $\theta$, we may reduce to:
\begin{equation}
I^2=\pi\int_{0}^{\infty}\,dr\,e^{-\frac{1}{2}u} = \dots
\end{equation}
the convolution of $e^{au}$ evaluates to $1/ae^{au}$ and we get:
\begin{equation}\label{eq:exconv}
I^2=\left. -2\pi e^{-\frac{1}{2}u}\right|_0^{\infty} = 2\pi
\end{equation}
and finally:
\begin{framed}
\begin{equation}
I=\sqrt{2\pi}
\end{equation}
\end{framed}

\subsection{(b)}
The steps from above also apply here. We just have to apply the rule used for eq. \ref{eq:exconv}. We will achieve:
\begin{equation}
I^2=\left. -(2/a)\pi e^{-\frac{1}{2}u}\right|_0^{\infty} = (2/a)\pi
\end{equation}
and finally:
\begin{framed}
\begin{equation}
I=\sqrt{(2/a)\pi}
\end{equation}
\end{framed}

\subsection{(c)}
\paragraph{c.1}

\begin{equation}
I=\int_{-\infty}^{\infty}\,dx\,xe^{-\frac{x^2}{2}}
\end{equation} 

performing a variable transform from $(x^2/2) \rightarrow u$ we get:
\begin{framed}
\begin{equation}\label{eq:xgauss}
I=\int_{u(-\infty)}^{u(\infty)}\,du\,e^{-u} = \left. -e^{-u}\right|_\infty^\infty=0
\end{equation} 
\end{framed}

\paragraph{c.2}
\begin{equation}
I=\int_{-\infty}^{\infty}\,dx\,x^2e^{-\frac{x^2}{2}}
\end{equation} 

We may write $I$ as:
\begin{equation}
I=\int_{-\infty}^{\infty}\,dx\,x \cdot xe^{-\frac{x^2}{2}}
\end{equation} 
we may use partial integration:
\begin{align}
I&=\left. x \left(\int\,dx\,xe^{-\frac{x^2}{2}}\right)\right|_{-\infty}^\infty+\int_{-\infty}^{\infty}\,dx\,e^{-\frac{x^2}{2}}\\
 &\overset{(\ref{eq:xgauss})}{=}\cancelto{0}{\left.xe^{-\frac{x^2}{2}}\right|_{-\infty}^\infty}+\int_{-\infty}^{\infty}\,dx\,e^{-\frac{x^2}{2}}
\end{align}
Which simply is the Gaussian integral from eq. \ref{eq:stdgauss}. Hence
\begin{framed}
\begin{equation}
I=\sqrt{2\pi}
\end{equation}
\end{framed} 

\subsection{(d)}

\begin{equation}
I=\int_{-\infty}^\infty\,dx\,e^{-\frac{x^2}{2}-bx}
\end{equation}
we will start by completing the square in the exponential function argument: 
\begin{equation}
I=\int_{-\infty}^\infty\,dx\,e^{-\frac{1}{2}(x^2+2bx+b^2)+\frac{1}{2}b^2}=\int_{-\infty}^\infty\,dx\,e^{-\frac{1}{2}(x+b)^2}\times e^{\frac{1}{2}b^2}
\end{equation}
Now we do a variable transform $x+b \rightarrow u$ and we get:
\begin{framed}
\begin{equation}
I=e^{\frac{1}{2}b^2}\times\int_{-\infty}^\infty\,dx\,e^{-\frac{1}{2}u^2}=e^{\frac{1}{2}b^2}\times \sqrt{2\pi}
\end{equation}
\end{framed}

\section{Problem 2}
\emph{Assumption:} The Grand potential and the internal energy are homogeneous of degree 1.
\\ ${}$ \\
\emph{Theorem:} let $f(x_1,\dots,x_k)$ be differantiable and homogeneous of degree $m$:
\begin{equation}
f(\lambda x_1,\dots,\lambda x_k)=\lambda^mf(x_1,\dots,x_k)
\end{equation}
Then it follows:
\begin{equation}
\frac{\partial f}{\partial x_1}x_1+\cdots+\frac{\partial f}{\partial x_k}x_k=mf
\end{equation}
\subsection{(a)}
\emph{show:} $U=TS-PV+\mu N$
The state of a system, described by the internal energy, is in an equilibrium state, which means, that the state variables are constant. We first apply the Theorem from above to $U(V,S,N)$ and get:
\begin{equation}
\frac{\partial U}{\partial V}_{S,N}V+\frac{\partial U}{\partial S}_{V,N}S+\frac{\partial U}{\partial N}_{V,S}N
\end{equation}
Now we may replace the partial derivatives and achieve:
\begin{framed}
\begin{equation}
U=-PV+TS+\mu N
\end{equation}
\end{framed}
\emph{show:} $\Omega=-PV$
We start with the definition of the Grand potential, which is:
\begin{equation}
\Omega=U-TS-\mu N
\end{equation}
Now we compare the total derivative of the potential to the one of the internal energy:
\begin{align}
dU &= -PdV-VdP+TdS+SdT+\mu dN+Nd\mu\\
d\Omega&=dU-TdS-SdT-\mu dN-Nd\mu=-PdV-VdP
\end{align}
From this we can follow, that the potential can only depend on the volume and the pressure.
\\ ${}$ \\
DUNNO FURTHER, THINK MORE!!! 

\end{document}


